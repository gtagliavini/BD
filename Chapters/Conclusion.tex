\chapter{Conclusion \& Future Work}\label{chap:conclusion}
\section{Conclusion}\label{sec:res_sum}
In this dissertation we designed, implemented, and validated a virtual prototype of a Phase-Change-Memory-based analog in-memory computing accelerator within the GVSoC simulation environment for the PULP platform. 
The core of the work is a C++ model that performs efficient matrix-vector multiplication while closely reproducing PCM device physics, including the analog resistance states and the associated read/write dynamics. 
This model was seamlessly integrated into GVSoC as a memory-mapped peripheral, which allows full-system simulation and enables the accelerator's performance to be evaluated within a complete system-on-chip environment. 
To improve simulation speed and scalability, we flattened data-structures to optimise memory-access patterns, reducing cache misses and increasing overall throughput, and we introduced multithreaded kernels that parallelise the MVM computations across cores. 
Extensive benchmarking identified the optimal thread counts for a variety of hardware configurations. 
The final accelerator model is highly configurable: users can adjust matrix sizes, cell sizes, and threading options, making the prototype a valuable tool for exploring architectural trade-offs in analog in-memory computing systems.
\section{Future Work}\label{sec:future_work}
While this dissertation has laid a solid foundation, there are several ways for future work that can further enhance the PCM-based AIMC accelerator model and its applications:\\
\paragraph{Extending the PCM Model:} Future work could focus on incorporating more detailed physical models of PCM devices,
  including variability, endurance, and retention characteristics. While the current model it's already capable of simulating drift resistance MVM by using the differential algorithm a well known effect of PCM devices
  is the variability of read operation while under thermal stress. The current model exposes read, write and computation counters that could be used for further analysis of these effects.\\
\paragraph{Exploring Additional Optimisation Techniques to Improve Simulation Performance:} Further research could investigate additional optimisation strategies.
  The current data-structure and the module itself could be further optimised by exploring both SIMD instructions by-design and GPU acceleration. 
  Without the need to move the PCM weights frequently, this module is a perfect candidate for GPU acceleration.\\
