
\thispagestyle{plain}
\begin{center}
    \Large
    \textbf{Exploring Phase-Change Memory as a Compute Accelerator Module:}
        
    \vspace{0.4cm}
    \large
    A Virtual Prototype on the PULP Platform
        
    \vspace{0.4cm}
    \textbf{Leonardo Domenicali}
       
    \vspace{0.9cm}
    \textbf{Abstract}
\end{center}

Traditional von Neumann architectures are hitting a wall. 
The physical separation between memory and processing creates a bottleneck that is becoming critical for data-intensive applications, 
particularly in AI and Machine Learning. 
Moving data back and forth between CPU and memory is becoming more time and energy consuming than the actual computation: this is the "memory wall" problem.

This dissertation explores Phase-Change Memory (PCM) as a potential solution through Analog In-Memory Computing (AIMC). 
The core idea is to perform the computation directly where the data is stored instead of relying on costly moving cycles. 
PCM is particularly interesting for this kind of application because its analog resistance states can naturally represent neural network weights, 
and its crossbar architecture allows for parallel computation of matrix-vector multiplications in a single analog operation.

The main contribution of this work is a working virtual prototype of a PCM-based AIMC accelerator. 
The model is implemented in C++ and integrated into the GVSoC simulation platform within the PULP ecosystem. 
A significant part of the work focused on optimising the simulation itself, making it sufficiently fast to be useful for experimentation. 
This involved designing cache-friendly data-structures and implementing multithreaded execution for the core Matrix-Vector Multiplication algorithm.
Performance benchmarks demonstrate the effectiveness of these optimisations. 
The flat buffer data-structure reduces cache misses by up to x times compared to a standard multidimensional array approach,  %TODO: re-check numbers again
and multithreading provides significant speedup on multi-core systems, with optimal performance achieved at 8 threads for the tested configurations.
The result is a configurable simulation framework that can be used to explore AIMC architectures and provides a 
foundation for co-designing hardware accelerators and the software needed to use it effectively.